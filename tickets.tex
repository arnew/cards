%%%
%
% ticket.sty example file for flashcards for word learning
%
%%%%
% use the corresponding paper size for your ticket definition
\documentclass[a4paper,10pt,oneside]{letter}


\usepackage[stratego,cutmark]{ticket}
\usepackage{graphicx}  % load misc stuff
\usepackage{syntonly}  %to test without output

\usepackage{yfonts}[1998/10/03]  % LaTeX interface to the blackletter (old german) fonts designed by Yannis Haralambous:  Gotisch, Schwabacher, Fraktur and the baroque initials. \textgoth{...}, \textswab{...},  \textfrak{...} and  \textinit{...} \gothfamily, \swabfamily, \frakfamily and \initfamily

\usepackage{fancybox}
\usepackage{rotating}

\usepackage{ifthen}

\usepackage[utf8]{inputenc}

\newcommand{\wb}[2]{\fontsize{#1}{#2}\usefont{U}{ornamente}{maeander}{n}}
% make your default ticket. \ticketdefault is somewhat like a background
\newcommand{\ticketdefaultfront}{%no background
\setlength{\unitlength}{1mm}%
%\put(0,0){{\wb{5mm}{5mm}%
%\multiput(5,95)(5,0){11}{\makebox(5,5)[t]{A}}%       Upper border
%\put(60,95){\makebox(5,5)[t]{F}}%                     Upper right corner
%\multiput(60,90)(0,-5){18}{\makebox(5,5)[t]{C}}%   Right border
%\put(60,0){\makebox(5,5)[t]{S}}%                  Lower right corner
%\multiput(5,0)(5,0){11}{\makebox(5,5)[t]{A}}%    Lower border
%\put(0,0){\makebox(5,5)[t]{N}}%                    Lower left corner
%\multiput(0,90)(0,-5){18}{\makebox(5,5)[t]{C}}%     Left border
%\put(0,95){\makebox(5,5)[t]{K}}%                       Upper left corner
%}}%
}

\newcommand{\ticketdefaultback}{
\setlength{\unitlength}{1mm}%
\put(0,0){{%
\fontsize{5mm}{5mm}%
\usefont{U}{ornamente}{maeander}{n}%
\put(0,95){LCDCDCDCDCDCH}%       Upper border
\multiput(0,90)(0,-10){9}{A}%     Left border
\multiput(0,85)(0,-10){9}{B}%     Left border
\put(0,0){PDCDCDCDCDCDT}%       Upper border
\multiput(60,90)(0,-10){9}{A}%     Left border
\multiput(60,85)(0,-10){9}{B}%     Left border
}}%
\put(5,5){{%
\fontsize{5mm}{5mm}%
\usefont{U}{ornamente}{floor}{n}%
\multiput(0,0)(0,10){9}{AB}%       Upper border
\multiput(0,5)(0,10){9}{BA}%       Upper border
\multiput(45,0)(0,10){9}{BA}%       Upper border
\multiput(45,5)(0,10){9}{AB}%       Upper border
\put(10,0){ABABABA}%       Upper border
\put(10,5){BABABAB}%       Upper border
\put(10,80){ABABABA}%       Upper border
\put(10,85){BABABAB}%       Upper border
}}%
}

% \strategocard{% staerker:
% }{% eigener rang:
% }{% schwaecher:
% }{% hinweise
% }{}

\newcommand{\strategocardfront}[5]{\ticket{% front
\put(5,5){\makebox(55,90){%
\begin{minipage}{55mm}%
\begin{center}%
%\vspace{\fill}%
\textswab{\fontsize{0.5cm}{0.5cm}\selectfont%
#1%
\vspace{5mm}%
{\fontsize{1.5cm}{1.5cm}\selectfont{#2}}\\%
\vspace{5mm}%
#3%
\vspace{10mm}%
#4%
}%
%\footnotesize{#5}%
\end{center}%
\end{minipage}%
}}%
}%
}

\newcommand{\makeinitial}[1]{{\textinit#1}}
\newcommand{\strategocardback}[5]{%
\ticket{%
\put(15,15){\framebox(35,70){%
\begin{minipage}{55mm}%
\begin{center}%
\begin{sideways}%
{\huge{\textswab{\textinit#2}}}%
\end{sideways}%
\end{center}%
\end{minipage}%
}%
}%
}%
}




%% you can generate this part from a database!
\begin{document}
\newcommand{\strategocard}[5]{\strategocardfront{#1}{#2}{#3}{#4}{#5}}
\renewcommand{\ticketdefault}{\ticketdefaultfront}
%
\strategocard{% staerker:
Bombe\\
Feldmarschall\\
}{% eigener rang:
General
}{% schwaecher:
Oberst\\
Major\\
Hauptmann\\
Minör\\
Leutnant\\
Spion\\
Unteroffizier\\
Fahne\\
}{
}{
}
%
%
\strategocard{% staerker:
}{% eigener rang:
Fahne
}{% schwaecher:
}{% hinweise
Darf nicht gefangen werden!  Ziel von Stratego ist, die Fahne des Gegners zu
ergattern.  Sei auf der Hut!  Dich kann jeder schlagen!
}{
1 maximal
}
%
\strategocard{% staerker:
Bombe\\
Feldmarschall\\
General\\
Oberst\\
Major\\
Hauptmann\\
Leutnant\\
}{% eigener rang:
Spion
}{% schwaecher:
Minör\\
Unteroffizier\\
Fahne\\
}{
Kann die Gefangenen im gegnerischen Lager befreien.
}{
2 mindestens
}
\strategocard{% staerker:
Bombe\\
Feldmarschall\\
General\\
Oberst\\
Major\\
Hauptmann\\
Leutnant\\
}{% eigener rang:
Spion
}{% schwaecher:
Minör\\
Unteroffizier\\
Fahne\\
}{
Kann die Gefangenen im gegnerischen Lager befreien.
}{
2 mindestens
}
%
\strategocard{% staerker:
Minör\\
}{% eigener rang:
Bombe
}{% schwaecher:
Feldmarschall\\
General\\
Oberst\\
Major\\
Hauptmann\\
Leutnant\\
Spion\\
Unteroffizier\\
Fahne\\
}{% hinweise
}{1:1 mit Minör}
%
\strategocard{% staerker:
Feldmarschall\\
General\\
Oberst\\
Major\\
Hauptmann\\
Spion\\
}{% eigener rang:
Minör
}{% schwaecher:
Bombe\\
Leutnant\\
Unteroffizier\\
Fahne\\
}{% hinweise
}{1:1 mit Bombe}
%
%
\strategocard{% staerker:
Bombe\\
General\\
Oberst\\
Major\\
Hauptmann\\
Minör\\
Leutnant\\
Spion\\
}{% eigener rang:
Unteroffizier
}{% schwaecher:
Feldmarschall\\
Fahne\\
}{% hinweise
}{1:1 mit Feldmarschall}
%
\strategocard{% staerker:
Bombe\\
Unteroffizier\\
}{% eigener rang:
Feldmarschall
}{% schwaecher:
General\\
Oberst\\
Major\\
Hauptmann\\
Minör\\
Leutnant\\
Spion\\
Fahne\\
}{% hinweise
}{1:1 mit Unteroffizier}
%
%
%
\strategocard{% staerker:
Bombe\\
Feldmarschall\\
General\\
}{% eigener rang:
Oberst
}{% schwaecher:
Major\\
Hauptmann\\
Minör\\
Leutnant\\
Spion\\
Unteroffizier\\
Fahne\\
}{% hinweise
}{evtl. nicht besetzen}
\strategocard{% staerker:
Bombe\\
Feldmarschall\\
General\\
Oberst\\
}{% eigener rang:
Major
}{% schwaecher:
Hauptmann\\
Minör\\
Leutnant\\
Spion\\
Unteroffizier\\
Fahne\\
}{% hinweise
}{evtl. nicht besetzen}
\strategocard{% staerker:
Bombe\\
Feldmarschall\\
General\\
Oberst\\
Major\\
}{% eigener rang:
Hauptmann
}{% schwaecher:
Minör\\
Leutnant\\
Spion\\
Unteroffizier\\
Fahne\\
}{% hinweise
}{evtl. nicht besetzen}
\strategocard{% staerker:
Bombe\\
Feldmarschall\\
General\\
Oberst\\
Major\\
Hauptmann\\
Minör\\
}{% eigener rang:
Leutnant
}{% schwaecher:
Spion\\
Unteroffizier\\
Fahne\\
}{% hinweise
}{evtl. nicht besetzen}


\backside%
\oddsidemargin=9.5mm
\renewcommand{\strategocard}[5]{\strategocardback{#1}{#2}{#3}{#4}{#5}}
\renewcommand{\ticketdefault}{\ticketdefaultback}
%
\strategocard{% staerker:
Bombe\\
Feldmarschall\\
}{% eigener rang:
General
}{% schwaecher:
Oberst\\
Major\\
Hauptmann\\
Minör\\
Leutnant\\
Spion\\
Unteroffizier\\
Fahne\\
}{
}{
}
%
%
\strategocard{% staerker:
}{% eigener rang:
Fahne
}{% schwaecher:
}{% hinweise
Darf nicht gefangen werden!  Ziel von Stratego ist, die Fahne des Gegners zu
ergattern.  Sei auf der Hut!  Dich kann jeder schlagen!
}{
1 maximal
}
%
\strategocard{% staerker:
Bombe\\
Feldmarschall\\
General\\
Oberst\\
Major\\
Hauptmann\\
Leutnant\\
}{% eigener rang:
Spion
}{% schwaecher:
Minör\\
Unteroffizier\\
Fahne\\
}{
Kann die Gefangenen im gegnerischen Lager befreien.
}{
2 mindestens
}
\strategocard{% staerker:
Bombe\\
Feldmarschall\\
General\\
Oberst\\
Major\\
Hauptmann\\
Leutnant\\
}{% eigener rang:
Spion
}{% schwaecher:
Minör\\
Unteroffizier\\
Fahne\\
}{
Kann die Gefangenen im gegnerischen Lager befreien.
}{
2 mindestens
}
%
\strategocard{% staerker:
Minör\\
}{% eigener rang:
Bombe
}{% schwaecher:
Feldmarschall\\
General\\
Oberst\\
Major\\
Hauptmann\\
Leutnant\\
Spion\\
Unteroffizier\\
Fahne\\
}{% hinweise
}{1:1 mit Minör}
%
\strategocard{% staerker:
Feldmarschall\\
General\\
Oberst\\
Major\\
Hauptmann\\
Spion\\
}{% eigener rang:
Minör
}{% schwaecher:
Bombe\\
Leutnant\\
Unteroffizier\\
Fahne\\
}{% hinweise
}{1:1 mit Bombe}
%
%
\strategocard{% staerker:
Bombe\\
General\\
Oberst\\
Major\\
Hauptmann\\
Minör\\
Leutnant\\
Spion\\
}{% eigener rang:
Unteroffizier
}{% schwaecher:
Feldmarschall\\
Fahne\\
}{% hinweise
}{1:1 mit Feldmarschall}
%
\strategocard{% staerker:
Bombe\\
Unteroffizier\\
}{% eigener rang:
Feldmarschall
}{% schwaecher:
General\\
Oberst\\
Major\\
Hauptmann\\
Minör\\
Leutnant\\
Spion\\
Fahne\\
}{% hinweise
}{1:1 mit Unteroffizier}
%
%
%
\strategocard{% staerker:
Bombe\\
Feldmarschall\\
General\\
}{% eigener rang:
Oberst
}{% schwaecher:
Major\\
Hauptmann\\
Minör\\
Leutnant\\
Spion\\
Unteroffizier\\
Fahne\\
}{% hinweise
}{evtl. nicht besetzen}
\strategocard{% staerker:
Bombe\\
Feldmarschall\\
General\\
Oberst\\
}{% eigener rang:
Major
}{% schwaecher:
Hauptmann\\
Minör\\
Leutnant\\
Spion\\
Unteroffizier\\
Fahne\\
}{% hinweise
}{evtl. nicht besetzen}
\strategocard{% staerker:
Bombe\\
Feldmarschall\\
General\\
Oberst\\
Major\\
}{% eigener rang:
Hauptmann
}{% schwaecher:
Minör\\
Leutnant\\
Spion\\
Unteroffizier\\
Fahne\\
}{% hinweise
}{evtl. nicht besetzen}
\strategocard{% staerker:
Bombe\\
Feldmarschall\\
General\\
Oberst\\
Major\\
Hauptmann\\
Minör\\
}{% eigener rang:
Leutnant
}{% schwaecher:
Spion\\
Unteroffizier\\
Fahne\\
}{% hinweise
}{evtl. nicht besetzen}

\end{document}
